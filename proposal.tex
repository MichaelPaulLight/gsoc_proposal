% Options for packages loaded elsewhere
\PassOptionsToPackage{unicode}{hyperref}
\PassOptionsToPackage{hyphens}{url}
\PassOptionsToPackage{dvipsnames,svgnames,x11names}{xcolor}
%
\documentclass[
  letterpaper,
  DIV=11,
  numbers=noendperiod]{scrartcl}

\usepackage{amsmath,amssymb}
\usepackage{iftex}
\ifPDFTeX
  \usepackage[T1]{fontenc}
  \usepackage[utf8]{inputenc}
  \usepackage{textcomp} % provide euro and other symbols
\else % if luatex or xetex
  \usepackage{unicode-math}
  \defaultfontfeatures{Scale=MatchLowercase}
  \defaultfontfeatures[\rmfamily]{Ligatures=TeX,Scale=1}
\fi
\usepackage{lmodern}
\ifPDFTeX\else  
    % xetex/luatex font selection
\fi
% Use upquote if available, for straight quotes in verbatim environments
\IfFileExists{upquote.sty}{\usepackage{upquote}}{}
\IfFileExists{microtype.sty}{% use microtype if available
  \usepackage[]{microtype}
  \UseMicrotypeSet[protrusion]{basicmath} % disable protrusion for tt fonts
}{}
\makeatletter
\@ifundefined{KOMAClassName}{% if non-KOMA class
  \IfFileExists{parskip.sty}{%
    \usepackage{parskip}
  }{% else
    \setlength{\parindent}{0pt}
    \setlength{\parskip}{6pt plus 2pt minus 1pt}}
}{% if KOMA class
  \KOMAoptions{parskip=half}}
\makeatother
\usepackage{xcolor}
\setlength{\emergencystretch}{3em} % prevent overfull lines
\setcounter{secnumdepth}{-\maxdimen} % remove section numbering
% Make \paragraph and \subparagraph free-standing
\ifx\paragraph\undefined\else
  \let\oldparagraph\paragraph
  \renewcommand{\paragraph}[1]{\oldparagraph{#1}\mbox{}}
\fi
\ifx\subparagraph\undefined\else
  \let\oldsubparagraph\subparagraph
  \renewcommand{\subparagraph}[1]{\oldsubparagraph{#1}\mbox{}}
\fi


\providecommand{\tightlist}{%
  \setlength{\itemsep}{0pt}\setlength{\parskip}{0pt}}\usepackage{longtable,booktabs,array}
\usepackage{calc} % for calculating minipage widths
% Correct order of tables after \paragraph or \subparagraph
\usepackage{etoolbox}
\makeatletter
\patchcmd\longtable{\par}{\if@noskipsec\mbox{}\fi\par}{}{}
\makeatother
% Allow footnotes in longtable head/foot
\IfFileExists{footnotehyper.sty}{\usepackage{footnotehyper}}{\usepackage{footnote}}
\makesavenoteenv{longtable}
\usepackage{graphicx}
\makeatletter
\def\maxwidth{\ifdim\Gin@nat@width>\linewidth\linewidth\else\Gin@nat@width\fi}
\def\maxheight{\ifdim\Gin@nat@height>\textheight\textheight\else\Gin@nat@height\fi}
\makeatother
% Scale images if necessary, so that they will not overflow the page
% margins by default, and it is still possible to overwrite the defaults
% using explicit options in \includegraphics[width, height, ...]{}
\setkeys{Gin}{width=\maxwidth,height=\maxheight,keepaspectratio}
% Set default figure placement to htbp
\makeatletter
\def\fps@figure{htbp}
\makeatother

\usepackage{booktabs}
\usepackage{caption}
\usepackage{longtable}
\usepackage{colortbl}
\usepackage{array}
\usepackage{anyfontsize}
\usepackage{multirow}
\KOMAoption{captions}{tableheading}
\makeatletter
\@ifpackageloaded{caption}{}{\usepackage{caption}}
\AtBeginDocument{%
\ifdefined\contentsname
  \renewcommand*\contentsname{Table of contents}
\else
  \newcommand\contentsname{Table of contents}
\fi
\ifdefined\listfigurename
  \renewcommand*\listfigurename{List of Figures}
\else
  \newcommand\listfigurename{List of Figures}
\fi
\ifdefined\listtablename
  \renewcommand*\listtablename{List of Tables}
\else
  \newcommand\listtablename{List of Tables}
\fi
\ifdefined\figurename
  \renewcommand*\figurename{Figure}
\else
  \newcommand\figurename{Figure}
\fi
\ifdefined\tablename
  \renewcommand*\tablename{Table}
\else
  \newcommand\tablename{Table}
\fi
}
\@ifpackageloaded{float}{}{\usepackage{float}}
\floatstyle{ruled}
\@ifundefined{c@chapter}{\newfloat{codelisting}{h}{lop}}{\newfloat{codelisting}{h}{lop}[chapter]}
\floatname{codelisting}{Listing}
\newcommand*\listoflistings{\listof{codelisting}{List of Listings}}
\makeatother
\makeatletter
\makeatother
\makeatletter
\@ifpackageloaded{caption}{}{\usepackage{caption}}
\@ifpackageloaded{subcaption}{}{\usepackage{subcaption}}
\makeatother
\ifLuaTeX
  \usepackage{selnolig}  % disable illegal ligatures
\fi
\usepackage{bookmark}

\IfFileExists{xurl.sty}{\usepackage{xurl}}{} % add URL line breaks if available
\urlstyle{same} % disable monospaced font for URLs
\hypersetup{
  pdftitle={Google Summer of Code Proposal: bayesplot},
  colorlinks=true,
  linkcolor={blue},
  filecolor={Maroon},
  citecolor={Blue},
  urlcolor={Blue},
  pdfcreator={LaTeX via pandoc}}

\title{Google Summer of Code Proposal: bayesplot}
\author{}
\date{}

\begin{document}
\maketitle

\section{About}\label{about}

In May, I'll finish a Master of Applied Data Science at the University
of Michigan; the curriculum has been Python-centric and essentially
acted as a crash course in vanilla machine learning, but provided some
foundations in causal inference, uncertainty, and math that have carried
over to my personal study of Bayesian stats. I'm motivated by the
prospect of making the Baysian workflow accessible to anyone who's
interested in applying it, and think bayesplot is a really key component
of both learning and maintaining a principled way of doing Bayesian
inference in the real world.

\begin{itemize}
\item
  \href{https://github.com/MichaelPaulLight}{Github}
\item
  \href{https://www.linkedin.com/in/michael-paul-light/}{Linkedin}
\item
  \href{michaelpaullight@gmail.com}{Email}
\end{itemize}

\section{Abstract}\label{abstract}

Bayesplot is an R package that provides an extensive library of plotting
functions for use after fitting Bayesian models (typically with MCMC).
The plots created by \textbf{bayesplot} are ggplot objects, which means
that after a plot is created it can be further customized using various
functions from the \textbf{ggplot2} package.

Currently \textbf{bayesplot} offers a variety of plots of posterior
draws, visual MCMC diagnostics, graphical posterior (or prior)
predictive checking, and general plots of posterior (or prior)
predictive distributions.

The goal of this project will be to implement the ensemble of visual
predictive checks proposed by
\href{https://teemusailynoja.github.io/visual-predictive-checks/}{Säilynoja}
et al.~in bayesplot, with a particular focus on adding visualizations
for predictive checks of discrete and categorical outcomes.

Visualizations to be added include:

\begin{itemize}
\item
  Overlaid KDE Plots
\item
  Discrete Rootograms
\item
  PAV-Adjusted Calibration Plots
\item
  PAV-Adjusted Residual Plots
\end{itemize}

Other project outcomes include documentation updates.

\section{Previous Contributions to
Stan}\label{previous-contributions-to-stan}

\subsection{Projpred}\label{projpred}

\begin{itemize}
\item
  \href{https://github.com/stan-dev/projpred/pull/509}{PR \#509}

  \begin{itemize}
  \tightlist
  \item
    Add contribution guidelines to README.Rmd
  \end{itemize}
\end{itemize}

\section{List of Deliverables}\label{list-of-deliverables}

Editable
\href{https://docs.google.com/spreadsheets/d/1llapJAsr9QqkNebSWEMnm0b4QB_sRLwOjwWQdF92Kak/edit?gid=1289123949\#gid=1289123949}{here}.

\begin{table}
\fontsize{12.0pt}{14.4pt}\selectfont
\begin{tabular*}{\linewidth}{@{\extracolsep{\fill}}lcc}
\toprule
deliverable & deadline & notes \\ 
\midrule\addlinespace[2.5pt]
Overlaid KDE Plots & NA & NA \\ 
Discrete Rootogram & NA & NA \\ 
PAV-Adjusted Calibration Plots & NA & NA \\ 
PAV-Adjusted Residual PlotsTimeline & NA & NA \\ 
\bottomrule
\end{tabular*}
\end{table}

\section{Timeline}\label{timeline}

Editable
\href{https://docs.google.com/spreadsheets/d/1llapJAsr9QqkNebSWEMnm0b4QB_sRLwOjwWQdF92Kak/edit?gid=0\#gid=0}{here}.

\begin{table}
\fontsize{12.0pt}{14.4pt}\selectfont
\begin{tabular*}{\linewidth}{@{\extracolsep{\fill}}rl}
\toprule
date & task \\ 
\midrule\addlinespace[2.5pt]
Thu, May 1, 2025 & Project start \\ 
Fri, May 16, 2025 & Proposal revised. Project deliverables list finalized. \\ 
Mon, Jun 2, 2025 & Timeline finalized. Work on deliverable 1 begun. \\ 
Fri, Jun 13, 2025 & NA \\ 
Fri, Jul 4, 2025 & NA \\ 
Fri, Jul 18, 2025 & NA \\ 
Fri, Aug 1, 2025 & NA \\ 
Sun, Aug 17, 2025 & NA \\ 
\bottomrule
\end{tabular*}
\end{table}

\section{Contributor/Project Fit}\label{contributorproject-fit}

\subsection{Motivation}\label{motivation}

The Stan team has made a huge imprint on my work and how I see the
world, and I'd love to contribute to the project in any way that I can.
I consider GSOC an opportunity to learn a ton and help get Stan to more
people. I'm motivated by the prospect of making the Baysian workflow
accessible to anyone who's interested in applying it, and think
bayesplot is a really key component of both learning and maintaining a
principled way of doing Bayesian inference in the real world.

I also think working on a GSOC project with the Stan team represents an
invaluable educational experience. I'm excited to learn new ways to
solve difficult problems and expand my capabilities as a data scientist
and software engineer.

\subsection{General Competencies}\label{general-competencies}

\subsubsection{R}\label{r}

\begin{itemize}
\item
  Used for personal and professional data projects since 2020. Lots of
  reps with packages in the tidyverse and tidymodels ecosystems, web
  scraping, and medium-big data processing tools like DuckDB/Arrow.
\item
  I write R code every day to do data science, but there's a lot of room
  for growth in my ability to produce code in production.
\end{itemize}

\subsubsection{Stan}\label{stan}

\begin{itemize}
\item
  I started learning Stan syntax within the past year. I read Stan
  models every day, and can write simple models. Most of my exposure to
  Stan is via brms; the more complex models I've built have been with
  brms.
\item
  I have a solid understanding of the theory behind prior, retrodictive,
  and predictive checks in the Bayesian workflow, and experience using
  bayesplot, tidybayes, ggdist, marginaleffects, and vmc to
  analyze/visualize models.
\item
  I've recently incorporated sbc into my workflow and have read the
  papers by Talts, Mondrak, and Bürkner. I get the theoretical need for
  sbc, can run sbc and begin to interpret its outputs, and am actively
  developing my understanding of the associated math.
\end{itemize}

\subsubsection{Bayes}\label{bayes}

\begin{itemize}
\tightlist
\item
  I started learning probability theory and Bayesian inference about a
  year ago, primarily using Statistical Rethinking, BDA3, the Bayesian
  Workflow paper, and Michael Betancourt's writing as resources. I took
  a one-day course on hierarchical modeling from Betancourt earlier this
  year. I have a solid grasp of the steps in the Bayesian workflow and
  their justification. I understand and can implement a justifiable
  version of the workflow to solve problems using observational data.
\end{itemize}

\subsubsection{Git}\label{git}

\begin{itemize}
\item
  I use git to organize all my personal and academic projects. I'm
  familiar with git best practices and can collaborate with others on a
  shared repository.
\item
  I have opened issues on open source projects, but these cases have
  been limited to fixing typos in documentation.
\end{itemize}

\section{Availability}\label{availability}

I'm available through the entire GSOC period, and plan on working on
this project at least 20 hours each week. I'll be happy to work with the
Stan team to adjust the total working hours as needed.

\section{Other GSOC Applications}\label{other-gsoc-applications}

I'm not applying for any other GSOC projects.

\section{After GSOC}\label{after-gsoc}

I hope to continue contributing to the broader Stan project as I figure
out the next steps in my professional or academic life.



\end{document}
